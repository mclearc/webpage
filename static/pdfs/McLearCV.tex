%% orignal template by Mark Eli Kalderon 

\documentclass[11pt]{article}

%%%------------------------------------------------------------------------
%%% Metadata
%%%------------------------------------------------------------------------

%% Change as needed. Or just add me as a coauthor. Only some of these are 
%% used below in the hyperref declaration and address banner section.
\def\myauthor{Colin McLear}
\def\mytitle{Vita}
\def\mycopyright{\myauthor}
\def\mykeywords{}
\def\mybibliostyle{plain}
\def\mybibliocommand{}
\def\mysubtitle{}
\def\myaffiliation{University of Nebraska--Lincoln}
\def\myaddress{Department of Philosophy}
\def\myemail{mclearc@gmail.com}
\def\myweb{http://colinmclear.net}
\def\myphone{607 216 8718}
\def\myfax{402 472 0626}
\def\myversion{}
\def\myrevision{}


\date{} % not used (revision control instead)
\def\mykeywords{Colin McLear, McLear CV, Resume, Philosophy}

%%%------------------------------------------------------------------------  
%%% Git version tracking 
%%%------------------------------------------------------------------------

%% If you don't use git or the vc package (from CTAN), comment this out.
%% If you comment it out, be sure to remove the \rfoot comment below, too.
% \input{vc}

%%%------------------------------------------------------------------------
%%% Required style files
%%%------------------------------------------------------------------------
\usepackage{url,fancyhdr}
%%\usepackage{revnum} % for reverse-numbered publications (revnumerate environment) if needed.

%% needed for xelatex to work
\usepackage{fontspec}
\usepackage{xunicode}

%% color for the links 
\usepackage{color}

%% hyperlinks
\usepackage[xetex, 
	colorlinks=true,
	urlcolor=blue,
	plainpages=false,
  	pdfpagelabels,
  	bookmarksnumbered,
  	pdftitle={\mytitle},
  	pagebackref,
  	pdfauthor={\myauthor},
  	pdfkeywords={\mykeywords}
  	]{hyperref}

%%%------------------------------------------------------------------------
%%% Document
%%%------------------------------------------------------------------------
\begin{document}

%% Choose fonts for use with xelatex
%% Minion and Myriad are widely available, from Adobe. 
%% Pragmata is available to buy at http://www.fsd.it/fonts/pragma.htm
%% and is worth every penny. Any good monospace font will work fine, though.
%% Consolas or inconsolata are good alternatives.
\setromanfont[Mapping={tex-text},Numbers={OldStyle},Ligatures={Common}]{Adobe Caslon Pro} 
\setsansfont[Mapping=tex-text,Colour=AA0000]{Optima}
\setmonofont[Mapping=tex-text,Scale=0.9]{Inconsolata} 


%%%------------------------------------------------------------------------
%%% Local commands
%%%------------------------------------------------------------------------

%% Marginal header
%% Note: as the document goes on you may need to introduce a (gradually increasing)
%% \vspace element to keep the marginal header pleasingly aligned with the first 
%% item in the body text. Like this: \marginhead{{\vskip 0.4em}Grants}, or 
%% \marginhead{{\vskip 0.8em}Service}. Experiment as needed.
\newcommand{\marginhead}[1]{\marginpar{\textsf{{\footnotesize\vspace{-1em}\flushright #1}}}}


%% custom ampersand (font consistent with the one chosen above)
% \newcommand{\amper}{{\fontspec[Scale=.95,Colour=AA0000]{Minion Pro Medium}\selectfont\&\,}}

%% No bullets on labels
\renewcommand{\labelitemi}{~} 

%% Custom hanging indent for vita items
\def\ind{\hangindent=1 true cm\hangafter=1 \noindent}
%\def\ind{\hangindent=18pt\hangafter=1 \noindent}
\def\labelitemi{~}
\renewcommand{\labelitemii}{~}

%%%------------------------------------------------------------------------
%%% Page layout
%%%------------------------------------------------------------------------
\pagestyle{fancy}
\renewcommand{\headrulewidth}{0pt}
\fancyhead{}
\fancyfoot{}
%\rhead{{\scriptsize\thepage}}
\rfoot{{\footnotesize\thepage}}
%% git revision control footer 
% \rfoot{\texttt{\scriptsize \VCRevision\ on \VCDateTEX}} % git revision info inserted via external script -- see docs for vc package for details. comment out this line if you're not using vc, and also remove the \input{vc} line above.

%%%------------------------------------------------------------------------
%%% Address and contact block
%%%------------------------------------------------------------------------
\begin{minipage}[t]{2.95in}
 \flushright {\footnotesize
   \href{http://www.unl.edu/philosophy/welcome}{Department of
     Philosophy} \\ 1003 Oldfather Hall \\University of Nebraska--Lincoln, \\ \vspace{-0.04in} Lincoln, \textsc{ne 68588--0321}}
  
\end{minipage}
\hfill     
%\begin{minipage}[t]{0.0in}
% dummy (needed here)
%\end{minipage}
\hfill
\begin{minipage}[t]{1.7in}
  \flushright \footnotesize Phone: \myphone \\ 
  Fax: \myfax  \\ 
  {\footnotesize  \texttt{\href{mailto:\myemail}{\myemail}}} \\
  {\footnotesize \texttt{\href{\myweb}{\myweb}}}
\end{minipage}  


\medskip

%% Name 
\noindent{\huge {\textsc{Colin McLear}}}
\reversemarginpar

\medskip       

%% Appointments
\medskip
\marginhead{{\vskip 0.em}Appointments}

\noindent\textsc{University of Nebraska–Lincoln \vspace{0.01in}}

\ind 2013-15. Assistant Professor, Department of Philosophy.

\noindent\textsc{Cornell University \vspace{0.01in}}

\ind 2012-13. Lecturer, Department of Philosophy.      

\bigskip

%% Education
\marginhead{{\vskip 0.3em}Education}
\medskip

\noindent\textsc{Cornell University \vspace{0.01in}}

\ind September, 2012 (Defended). PhD, Philosophy.
\medskip

\noindent\textsc{Humboldt-Universität zu Berlin \vspace{0.01in}}

\ind 2011-12. DAAD, Visiting Scholar.

\medskip
\noindent\textsc{Heidelberg Universität\vspace{0.01in}}

\ind 2010-11. Visiting Student.

\medskip
\noindent\textsc{Kenyon College\vspace{0.02in}}

\ind May, 2000. BA, Philosophy, Music.

\bigskip
 
%% AOS/AOC

\marginhead{{\vskip 0.3em}AOS}
\medskip

\ind History of Modern (especially Kant), Philosophy of Mind.

\medskip

\marginhead{{\vskip 0.3em}AOC}
\medskip

\ind Aesthetics, Epistemology, Ethics, German Idealism, Political Theory.

\medskip

%% Publications


%%%% Book
%\marginhead{{\vskip 0.3em}Publications}
%\medskip
%\noindent\emph{Books \vspace{0.01in}}

%\ind  Mark Eli Kalderon. 2005. \emph{\href{http://ukcatalogue.oup.com/product/9780199275977.do}{Moral Fictionalism}}. Oxford:~Oxford University Press. \vspace{-0.075in}

%%%% Edited Volumes
%\medskip
%\noindent\emph{Edited Volumes \vspace{0.01in}}

%\ind Mark Eli Kalderon. 2005. \emph{\href{http://ukcatalogue.oup.com/product/9780199282180.do}{Fictionalism in Metaphysics}} Oxford:~Oxford University Press. \vspace{-0.075in}
 
\normalsize

\medskip

\marginhead{{\vskip 0.2em}Publications}
\medskip

%\noindent\textsc{Journal articles \vspace{0.05in}}
 
%% Use revnumerate environment if numbered publications are needed. 
%% (Include it above in the preamble).
%% \renewcommand{\labelenumi}{\textsc{a}\theenumi.}
%% \begin{revnumerate}

\ind Forthcoming. ``\href{https://www.dropbox.com/s/rpey031drd2low5/KantUnity.pdf}{Two Kinds of Unity in the \emph{Critique of Pure Reason}}.'' \emph{Journal of the History of Philosophy}.

\ind 2011. ``\href{http://quod.lib.umich.edu/cgi/p/pod/dod-idx?c=phimp;idno=3521354.0011.015}{Kant on Animal Consciousness}.'' \emph{Philosophers’ Imprint} 11(15):~1--16. 

\ind 2010. ``\href{http://onlinelibrary.wiley.com/doi/10.1111/j.1468-0149.2010.00513.x/abstract}{Three Skeptics and the Critique: Critical Notice of Michael Forster’s Kant and Skepticism}.'' Co-written with Andrew Chignell. 
 \emph{Philosophical Books} 51(4):~228--244.

\bigskip

%\end{revnumerate}
%\newpage
%\noindent\emph{Book chapters \vspace{0.05in}}
% \renewcommand{\labelenumi}{\textsc{c}\theenumi.}
% \begin{revnumerate}

%\end{revnumerate}

%\bigskip 

 
%\newpage

%\noindent\emph{In preparation \vspace{0.05in}}

%\renewcommand{\labelenumi}{\textsc{r}\theenumi.}
%\begin{revnumerate}

%\ind Mark Eli Kalderon. Forthcoming, \emph{Dialectica}. ``Color and the Problem of Perceptual Presence''. Like dispositional theories of color, Noë's phenomenal objectivism attempts to understand being colored in terms of looking colored. The account is meant to resolve a prima facie conflict in or between experiences in cases of color constancy. I argue that no prima facie conflict is generated and that phenomenal objectivism, like the dispositional theory, fails to provide an adequate account of color constancy.

% %\end{revnumerate}
% \bigskip

%% Dissertation

\marginhead{{\vskip 0.2em}Dissertation Abstract}
\medskip

\ind \textsc{Essays on Kant on Perception and Cognition}
\medskip

\noindent The dissertation discusses five issues in Kant’s theory of perception and cognition regarding (i) the nature of perception and its difference from hallucination; (ii) whether and what kind of representational capacities are needed for perception; (iii) how, with respect to Kant’s epistemological views, perceptual experience grounds or justifies basic empirical belief; (iv) the independence of sensibility from the understanding; (v) the role of the understanding in cognition. 

%\bigskip
\newpage

%% Presentations
\marginhead{{\vskip 0.3em}Presentations}
\medskip

\ind 2013. ``Objects and Objectivity in the Transcendental Deduction.'' Contemporary Kantian Philosophy Workshop, University of Luxembourg. October.

\ind 2013. ``Animals and Objectivity.'' Kant on Animals Conference, University of the Witwatersrand, South Africa. July.

\ind 2013. ``Kant and McDowell on Perceptual Givenness." Central division of the American Philosophical Association. February.

\ind 2013. ``Kant on the Representation of an Object.'' University of Michigan; University of Illinois, Urbana-Champaign; Illinois State University. January--Febuary. 

\ind 2012. ``Intellectualism and the Transcendental Deduction.’’ Eastern division of the American Philosophical Association. December.

\ind 2012. Comments on Desmond Hogan's ``Force Monism and Rationalism.'' Workshop on Force, Upstate New York Workshop in Early Modern Philosophy. December. 

\ind 2012. Comments on Andrew Specht, ``Rethinking the Neglected Alternative.'' Meeting of the Creighton Club: New York Philosophical Association, Hobart \& William Smith Colleges. November.

\ind 2012. ``Kant on Relation to an Object.'' Philosophy Workshop/Colloquium, Cornell University. November.

\ind 2012. ``Form, Matter, and Relation to an Object.’’ Kolloquium Prof. Dr. Tobias Rosefeldt, Humboldt-Universität zu Berlin. May.

\ind 2012. ``Two Conceptions of Unity in Kant’s Critique of Pure Reason.” Eastern Meeting of the North American Kant Society, Princeton University. April.

\ind 2012. “Kant and McDowell on Perceptual Givenness.” Workshop on the Epistemology of Perception, Birkbeck University. March

\ind 2011. “Kant and Perceptual Content,” Kolloquium Prof. Dr. Tobias Rosefeldt, Humboldt-Universität zu Berlin. November.

\ind 2011. ``Kant and Perceptual Content: A Plea for Austere Relationalism.” Conceptual
Content: History and Prospects, Cambridge University. September.

\ind 2011. Comments on Lawrence Pasternack’s “Kant on Opinion.” 1st Biannual Meeting of the North American Kant Society. University of Illinois, Urbana-Champaign. June.

\ind 2011. ``Strands of Subjectivity: Kant on the Presentation and Projection of Sensible 
Qualities.” Philosophy Workshop/Colloquium, Cornell University. February.

\ind 2011.  ``Kantian Aesthetics and the Problem of Sensory Communicability.” 
Sinn und Sinnlichkeit: Uses and Abuses of Aesthetics Today, Cornell University. February.

\ind 2010. ``Disjunctivism and Psychology.” Philosophy Workshop/Colloquium, Cornell University. May.

\ind 2010. Comments on Adam Pautz’s "Why Consciousness Cannot Just be in the Head: A 			New Argument against Biological Theories.” Cornell University. March.

\ind 2010. ``Descartes on Cognizing Material Particulars.” Upstate New York Early Modern 			Philosophy Workshop, Syracuse University. March.

\ind 2008. ``Descartes on Sensation: The Syntactic-Causation Model.” 4th Biennial 				Margaret Dauler Wilson Conference, Cornell University. June.

\ind 2008. ``Identity and Transcendental Idealism.” Graduate Prize Essay; East division of the North American Kant Society, City University of New York. April.

%\end{revnumerate}

\bigskip

%% Teaching

\marginhead{{\vskip 0.2em}Teaching}
\medskip

\ind \textsc{University of Nebraska--Lincoln:}
\medskip

\ind 2013. Assistant Professor -- Graduate Seminar on Conceptualism. Fall. 

\ind 2013. Assistant Professor -- Introduction to Philosophy. Fall.

\medskip

\ind \textsc{Cornell University:}
\medskip

\ind 2013. Lecturer -- Technology and Society. Spring.

\ind 2013. Lecturer -- Introduction to Philosophy. Spring. 

\ind 2012. Lecturer -- Politics and Human Nature. Fall.

\ind 2009. Teaching Assistant -- Science and Human Nature, with Richard Boyd. Spring

\ind 2008. Teaching Assistant -- Introduction to Ethics, with Terence Irwin. Fall.

\ind 2008. Teaching Assistant -- History of Modern, with Andrew Chignell. Spring.

\ind 2008. Grader -- Rationalism, with Andrew Chignell. Spring.

\ind 2007. Teaching Assistant -- Moral Problems, with Peter Sutton. Fall.

\ind 2006. Grader -- Kant’s Critical Philosophy, with Michelle Kosch. Spring.
\medskip

\noindent \textsc{Lakes Region Community College (Laconia, NH):}
\medskip

\ind 1999. Teaching Assistant – Introduction to Philosophy

\bigskip

%% Service

\marginhead{{\vskip 0.325em}Service to the \newline Profession}
\medskip

\ind Referee for \emph{Kantian Review}, \emph{Mind}, \emph{Philosophers' Imprint}, \emph{Social Theory and Practice}.

\ind 2013--. Subject-area Editor: Kant's Works, Philpapers.com

\ind 2013. Co-organizer NAKS Biannual Meeting.

\ind 2009. Philosophy Faculty Search Committee.

\ind 2006-8. Graduate Student Representative.

\bigskip

%% Languages

\marginhead{{\vskip .35em}Languages}
\medskip

\ind German (reading and speaking). 

\ind Latin (basic reading).

\newpage

%% Awards

\marginhead{{\vskip 0.4em}Awards}
\medskip

\ind 2012-13. Philosophical Review Lectureship.

\ind 2011-12. DAAD Research Grant.

\ind 2010-11. Cornell/Heidelberg Exchange Fellowship.

\ind 2006/2009. Sage Fellowship, Cornell University.

\ind 2000. Virgil C. Aldrich Prize, Kenyon College.

\bigskip


%\marginhead{{\vskip 0.4em}Conferences Organized}
% \medskip

%\ind 2005. \emph{Frege, Identity, and Logic}, Institute of Philosophy in conjunction with the Mind Association. Papers by Ben Caplan, Richard Heck, Mark Eli Kalderon, Ian Rumfitt, Mark Sainsbury, Tom Smith, and Mark Textor.

%\bigskip

%\marginhead{{\vskip 0.4em}Grants}
%\medskip
  
%\ind 2004. AHRB Research Leave Scheme.

%\bigskip 

%\marginhead{{\vskip 0.9em}Administration}
%\medskip

%\ind 2011--present. Affiliate Tutor.



%% References
%\newpage
\marginhead{{\vskip 0.3em}References}
\medskip

\ind \textsc{Dissertation Committee:}
\medskip

\ind Andrew Chignell, \href{mailto:chignell@cornell.edu}{chignell@cornell.edu} – Cornell University, Philosophy.
 
\ind Michelle Kosch, \href{mailto:mak229@cornell.edu}{mak229@cornell.edu} – Cornell University, Philosophy.
 
\ind Derk Pereboom, \href{mailto:dp346@cornell.edu}{dp346@cornell.edu} – Cornell University, Philosophy.

\ind Nico Silins, \href{mailto:ns338@cornell.edu}{ns338@cornell.edu} – Cornell University, Philosophy.

\medskip

\ind \textsc{External References:}
\medskip

\noindent Lucy Allais, \href{mailto:lucy.allais@wits.ac.za}{lucy.allais@wits.ac.za} – University of Witwatersrand/University of Sussex, Philosophy.

\noindent Lawrence Pasternack, \href{mailto:l.pasternack@okstate.edu}{l.pasternack@okstate.edu} – Oklahoma State University, Philosophy. 

\noindent Tobias Rosefeldt, \href{mailto:tobias.rosefeldt@hu-berlin.de}{tobias.rosefeldt@hu-berlin.de} – Humboldt-Universität zu Berlin, Philosophy.

\noindent Ted Sider, \href{mailto:sider@cornell.edu}{sider@cornell.edu} - Cornell University, Philosophy.
\end{document}
